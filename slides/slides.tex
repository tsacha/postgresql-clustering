\documentclass[utf8]{beamer}
\usepackage[utf8]{inputenc}
\usepackage[french]{babel}
\usetheme{Warsaw}
%\useoutertheme{infolines}

\title{Clustering de Bases de Données}
\author{S. Trémoureux, H. Blanchard, S. Vernotte}
\institute{IUT Nancy-Chalemagne}
\date{27 mars 2013}

\defbeamertemplate*{footline}{shadow theme}
{%
  \leavevmode%
  \hbox{\begin{beamercolorbox}[wd=.5\paperwidth,ht=2.5ex,dp=1.125ex,leftskip=.3cm plus1fil,rightskip=.3cm]{author in head/foot}%
    \usebeamerfont{author in head/foot}\insertframenumber\,/\,\inserttotalframenumber\hfill\insertshortauthor
  \end{beamercolorbox}%
  \begin{beamercolorbox}[wd=.5\paperwidth,ht=2.5ex,dp=1.125ex,leftskip=.3cm,rightskip=.3cm plus1fil]{title in head/foot}%
    \usebeamerfont{title in head/foot}\insertshorttitle%
  \end{beamercolorbox}}%
  \vskip0pt%
}


\begin{document}

\begin{frame}
  \titlepage
\end{frame}

\section*{Sommaire}
\begin{frame}
  \tableofcontents
\end{frame}      

\section{Le clustering — terminologie utile}

\subsection{Définitions}

\begin{frame}
  \frametitle{Termes courants}

  \begin{itemize}
  \item Actif/passif
  \item Actif/actif
  \item Journaux de transactions
  \item Warm standby
  \item Hot standby  
  \end{itemize}
\end{frame}

\begin{frame}
  \frametitle{Propriétés ACID}

  \begin{itemize}
  \item Atomicité
  \item Cohérence
  \item Isolation
  \item Durabilité
  \end{itemize}
\end{frame}

\subsection{Différents types de réplication}

\begin{frame}
  \frametitle{Réplication} 

  \begin{itemize}
  \item Définition
  \end{itemize}
  
  \begin{itemize}
  \item Réplication asynchrone
  \begin{itemize}
    \item Asymétrique
    \item Symétrique
  \end{itemize}
  \end{itemize}

  \begin{itemize}
  \item Réplication synchrone
  \begin{itemize}
    \item Asymétrique
    \item Symétrique
  \end{itemize}
    
  \end{itemize}
\end{frame}

\subsection{La haute disponibilité et le failover}
\begin{frame}
  \frametitle{La haute disponibilité et le failover}

  \begin{itemize}
  \item Cluster haute disponibilité
  \item Cluster à répartition de charge
  \end{itemize}
\end{frame}


\section{Bucardo}

\subsection{Identité}

\begin{frame}
  \frametitle{Identité}
  \begin{itemize}
  \item Projet né d'une demande de développement spécifique
  \item Outil développé par End Point au US
  \item Mis en production à partir de 2002
  \item Ré-écrit en 2006
  \item En fonction chez Backcountry depuis 2006
  \end{itemize}

\end{frame}

\begin{frame}
  \frametitle{Identité}
  \begin{itemize}
  \item Projet BSD depuis 2007 en version 3.0.6
  \item Greg Sabino Mulane est l'un des principaux contributeurs
  \item Bucardo en version 4.5 à ce jour
  \item Une version 4.99 préfigure une version 5 à venir
  \end{itemize}
\end{frame}

\subsection{Tour d'horizon}


\begin{frame}
  \frametitle{Fonctionnement}

  \begin{itemize}
  \item Système de réplication asynchrone – symétrique
  \item Granularité au niveau ligne de table ou séquence
  \item S'appuie sur les déclencheurs pour engendrer ses actions
  \item Consigne tous les changements au niveau ligne
  \item Pré-requis Bucardo sont communs
  \end{itemize}
\end{frame}
  

\begin{frame}
  \frametitle{Pré-requis techniques}

  \begin{itemize}
  \item Postgresql 8.3 ou ultérieur
  \item Perl et quelques packages, DBD-pg, DBI et DBIx-safe
  \item pl/pgsql intégré à Postgresql
  \item pl/perl-U intégré à Postgresql
  \end{itemize}
\end{frame}

\begin{frame}
  \frametitle{Mode d'action}

  \begin{itemize}
  \item Démons écrits en Perl
  \item Démon principal : contrôleur maître
  \item Contrôleurs secondaires par synchronisation
  \item Démons de mise en action des synchronisations
  \end{itemize}
\end{frame}

\begin{frame}
  \frametitle{Possibilités} 

  \begin{itemize}
  \item Synchronisation maître/esclave
  \item Usage de plusieurs esclaves, il est 'multi-slave'
  \item S'appuie sur des ordres de type 'fullcopy' ou 'pushdelta'
  \item Migration de bases envisageable et assez simple
  \item Pas de réplication de type 'Hot Standby'
  \end{itemize}
\end{frame}

\begin{frame}
  \frametitle{Possibilités (suite)} 

  \begin{itemize}
  \item Synchronisation maître/maître
  \item Ordres de type 'swap'
  \item Résolution standard des conflits d'écriture
  \item Pas de répartition de charge
  \end{itemize}
\end{frame}


\begin{frame}
  \frametitle{Possibilités (suite)} 

  \begin{itemize}
  \item Limité à deux noeuds en master/master
  \item Déployable qu'en environnement Unix
  \item Pas de réplication de requêtes DDL
  \item Pas de pooling de connexions
  \end{itemize}
\end{frame}


\begin{frame}
  \frametitle{Architecture} 

  \begin{itemize}
  \item Architecture flexible
  \item Maître, esclave et contrôleur Bucardo dans la même infrastructure
  \item Contrôleur et maître ou esclave dans des infrastructures différentes
  \end{itemize}
\end{frame}


\subsection{Installation et mise en route}

\begin{frame}
  \frametitle{Installation} 

  \begin{itemize}
  \item Intégration bucardo simplicime : bucardo\_ctl install
  \item Préparation du maître facile : bucardo\_ctl add (db \& all tables)
  \item Preparation de l'hôte esclave tout aussi facile
  \item Vérification des paramètres : bucardo\_ctl validate all
  \end{itemize}
\end{frame}



\begin{frame}
  \frametitle{Mise en route} 

  \begin{itemize}
  \item Création des set de réplication incroyablement intelligibles
  \item Type 'fullcopy' : bucardo\_ctl add sync copyall source= targetdb= type=
  \item Type 'pushdelta' : bucardo\_ctl add sync delta source= targetdb= type=
  \item Type 'swap' : bucardo\_ctl add sync 2ways source= targetdb= type=
  \item Puis bucardo\_ctl start et les triggers sont installés
  \end{itemize}
\end{frame}


\begin{frame}
  \frametitle{Maintenance} 

  \begin{itemize}
  \item langage de commande assez complet, accessible au niveau du shell
  \item bucardo\_ctl status
  \item bucardo\_ctl list db, herd, sync, dbgroup, tables
  \item bucardo\_ctl reload syncname
  \end{itemize}
\end{frame}


\subtitle{Intégration}

\begin{frame}
  \frametitle{Intégration} 
  \begin{itemize}
  \item Bucardo, HA et pooling de connexions
  \item Haproxy
  \item pgBouncer
  \item anti-spof, heartbeat
  \end{itemize}
\end{frame}


\section{pgPool}

\subtitle{Identité}

\begin{frame}
  \frametitle{Identité} 
  \begin{itemize}
  \item Middleware
  \item Licence BSD
  \item Développé par Tatsuo Ishii
  \item Grande longévité, développement actif
  \end{itemize}
\end{frame}

\subsection{Fonctionnalités}

\begin{frame}
  \frametitle{Fonctionnalités}
  
  Un grand nombre de fonctionnalités : 

  \begin{itemize}
  \item pooling de connexions
  \item load balancing
  \item parallélisation des requêtes
  \item failover et failback
  \end{itemize}
\end{frame}
  
\begin{frame}
  \begin{itemize}
  \item système de cache mémoire
  \item réplication native
  \item réplication maître/esclave : par Slony ou par streaming replication
  \end{itemize}
\end{frame}

\subsection{Réplication par transfert de journaux} 

\begin{frame}
  \frametitle{Réplication par transfert de journaux}

  \begin{itemize}
    \item Fonctionnalité récente (9.0) de PGSQL
    \item Utilisation des journaux de transactions
    \item Envoi aux serveurs esclaves (log-shipping)
    \item Connexion directe (streaming replication)
    \item Maître en « hot standby », esclave en « recovery »
  \end{itemize}
\end{frame}

\subsection{Configuration}

\begin{frame}
  \frametitle{Configuration de pgPool}

  \begin{itemize}
    \item Mode maître-esclave
    \item Au dessus d'une architecture déjà en place
    \item Philosophie UNIX
  \end{itemize}
\end{frame}

\section{Conclusion}

\begin{frame}
  \frametitle{Conclusion}

  \begin{itemize}
    \item Différentes solutions pour des approches différentes
    \item Logiciels pas si similaires
    \item Comparaison avec Oracle
    \item Et le libre dans tout ça ?
  \end{itemize}
\end{frame}

\section{Démonstrations}

\begin{frame}
  \frametitle{Démonstrations}

  \begin{itemize}
    \item Bucardo
    \item …pgpool
  \end{itemize}
\end{frame}

\end{document}
